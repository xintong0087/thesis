% T I T L E   P A G E
% -------------------
% Last updated June 14, 2017, by Stephen Carr, IST-Client Services
% The title page is counted as page `i' but we need to suppress the
% page number. Also, we don't want any headers or footers.
\pagestyle{empty}
\pagenumbering{roman}

% The contents of the title page are specified in the "titlepage"
% environment.
\begin{titlepage}
        \begin{center}
        \vspace*{1.0cm}

        \Huge
        {\bf Resilient Machine Learning Approaches for Fast Risk Evaluation and Management in Financial Portfolios and Variable Annuities}

        \vspace*{1.0cm}

        \normalsize
        by \\

        \vspace*{1.0cm}

        \Large
        Xintong Li \\

        \vspace*{3.0cm}

        \normalsize
        A thesis \\
        presented to the University of Waterloo \\ 
        in fulfillment of the \\
        thesis requirement for the degree of \\
        Doctor of Philosophy \\
        in \\
        Actuarial Science \\

        \vspace*{2.0cm}

        Waterloo, Ontario, Canada, 2025 \\

        \vspace*{1.0cm}

        \copyright\ Xintong Li, 2025 \\
        \end{center}
\end{titlepage}

% The rest of the front pages should contain no headers and be numbered using Roman numerals starting with `ii'
\pagestyle{plain}
\setcounter{page}{2}

\cleardoublepage % Ends the current page and causes all figures and tables that have so far appeared in the input to be printed.
% In a two-sided printing style, it also makes the next page a right-hand (odd-numbered) page, producing a blank page if necessary.

 
% E X A M I N I N G   C O M M I T T E E (Required for Ph.D. theses only)
% Remove or comment out the lines below to remove this page
\begin{center}\textbf{Examining Committee Membership}\end{center}
  \noindent
The following served on the Examining Committee for this thesis. The decision of the Examining Committee is by majority vote.
  \bigskip
  
%   \noindent
% \begin{tabbing}
% Internal-External Member: \=  \kill % using longest text to define tab length
% External Examiner: \>  John Smith \\ 
% \> Professor, Dept. of Philosophy of XX, University of ABC \\
% \end{tabbing} 
%   \bigskip
  
  \noindent
\begin{tabbing}
Internal-External Member: \=  \kill % using longest text to define tab length
Supervisor(s): \> Mingbin Feng \\
\> Associate Professor, Dept. of Statistics and Actuarial Science\\
\> University of Waterloo \\[1cm]
\> Tony S. Wirjanto \\
\> Professor, Dept. of Statistics and Actuarial Science \\
\> University of Waterloo \\
\end{tabbing}
  \bigskip
  
  \noindent
  \begin{tabbing}
Internal-External Member: \=  \kill % using longest text to define tab length
Internal Member: \> Mary R. Hardy \\
\> Professor, Dept. of Statistics and Actuarial Science \\
\> University of Waterloo\\[1cm]
\> Chengguo Weng \\
\> Professor, Dept. of Statistics and Actuarial Science \\
\> University of Waterloo \\
\end{tabbing}
  \bigskip
  
%   \noindent
% \begin{tabbing}
% Internal-External Member: \=  \kill % using longest text to define tab length
% Internal-External Member: \> John Smith \\
% \> Professor, Dept. of XX, University of ABC \\
% \end{tabbing}
%   \bigskip
  
%   \noindent
% \begin{tabbing}
% Internal-External Member: \=  \kill % using longest text to define tab length
% Other Member(s): \> John Smith \\
% \> Professor, Dept. of XX, University of ABC \\
% \end{tabbing}

\cleardoublepage

% D E C L A R A T I O N   P A G E
% -------------------------------
  % The following is a sample Delaration Page as provided by the GSO
  % December 13th, 2006.  It is designed for an electronic thesis.
  \noindent
I hereby declare that I am the sole author of this thesis. This is a true copy of the thesis, including any required final revisions, as accepted by my examiners.

  \bigskip
  
  \noindent
I understand that my thesis may be made electronically available to the public.

\cleardoublepage

% A B S T R A C T
% ---------------

\begin{center}\textbf{Abstract}\end{center}

Risk management of financial derivatives and actuarial products is intricate and often requires modelling the underlying stochasticity with Monte Carlo simulations.
Monte Carlo simulation is flexible, and it can easily adapt to changes in model assumptions and market conditions.
However, as multiple sources of risk are considered over long time horizons, the simulation model becomes complex and time-consuming to run. 
Tremendous research effort has been dedicated to the design of computationally efficient machine learning-based procedures that mitigate the computational burden of a standard simulation procedure.
In machine learning, model flexibility comes at the expense of model resilience, which is crucial for risk management tasks.
This study considers estimating tail risks of complex financial and actuarial products with resilient machine learning-based nested simulation procedures.
We propose a novel metamodeling approach that integrates deep neural networks within a nested simulation framework for efficient risk estimation.
Our approaches offer substantial improvements over the associated standard simulation procedures.
This study also illustrates how to build and assess resilient machine learning models for different problem complexities and different data structures, qualities, and quantities.
To further enhance adaptability to new variable annuity contracts and changing market conditions, this thesis explores transfer learning techniques. 
By reusing and fine-tuning pre-trained metamodels, the proposed approach accelerates the adaptation process to different contract features and evolving market dynamics without retraining models from scratch. 
Transfer learning improves computational efficiency and enhances the robustness and flexibility of neural network metamodels in dynamic hedging of variable annuities.

Extensive numerical experiments in this thesis demonstrate that the proposed methods substantially improve computational efficiency, sometimes shortening runtime by orders of magnitude compared to standard nested simulation procedures. 
The results indicate that the deep neural network metamodels with transfer learning can quickly adapt to new market scenarios and contract specifications.
This research contributes to the advancement of risk management practices for complex actuarial products and financial derivatives.
By leveraging advanced machine learning techniques, the thesis offers a practical and scalable solution for insurers to perform timely and accurate risk assessments.
The integration of long short-term memory metamodels and transfer learning into a nested simulation framework represents a major step forward toward more efficient, adaptable, and robust methodologies in actuarial science and quantitative finance.


\cleardoublepage

% A C K N O W L E D G E M E N T S
% -------------------------------

\begin{center}\textbf{Acknowledgements}\end{center}

I would like to express my deepest gratitude to Professor Mingbin Feng and Professor Tony Wirjanto for their invaluable support and guidance throughout my academic journey. 
Their unwavering patience and profound wisdom have been instrumental in helping me navigate the complexities of my research. 
I am deeply grateful for their mentorship, which has not only sharpened my analytical skills but also inspired me to pursue excellence in my work.

Additionally, I extend my sincere thanks to my best friends, who have stood by me during the most challenging times. 
Their unwavering support and encouragement have been a constant source of strength that has helped me overcome obstacles and maintain my motivation throughout this process. 
Words cannot fully convey how much their presence has meant to me over these years.

I also wish to acknowledge the support of my family, who have always been a source of love and encouragement.
They have provided me with the emotional stability I needed to focus on my studies and research.

\cleardoublepage

% T A B L E   O F   C O N T E N T S
% ---------------------------------
\renewcommand\contentsname{Table of Contents}
\tableofcontents
\cleardoublepage
\phantomsection    % allows hyperref to link to the correct page

% L I S T   O F   T A B L E S
% ---------------------------
\addcontentsline{toc}{chapter}{List of Tables}
\listoftables
\cleardoublepage
\phantomsection		% allows hyperref to link to the correct page

% L I S T   O F   F I G U R E S
% -----------------------------
\addcontentsline{toc}{chapter}{List of Figures}
\listoffigures
\cleardoublepage
\phantomsection		% allows hyperref to link to the correct page

% GLOSSARIES (Lists of definitions, abbreviations, symbols, etc. provided by the glossaries-extra package)
% -----------------------------
\printglossaries
\cleardoublepage
\phantomsection		% allows hyperref to link to the correct page

% Change page numbering back to Arabic numerals
\pagenumbering{arabic}

