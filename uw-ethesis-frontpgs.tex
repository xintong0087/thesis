% T I T L E   P A G E
% -------------------
% Last updated June 14, 2017, by Stephen Carr, IST-Client Services
% The title page is counted as page `i' but we need to suppress the
% page number. Also, we don't want any headers or footers.
\pagestyle{empty}
\pagenumbering{roman}

% The contents of the title page are specified in the "titlepage"
% environment.
\begin{titlepage}
        \begin{center}
        \vspace*{1.0cm}

        \Huge
        {\bf Resilient Machine Learning Approaches to Risk Evaluation and Hedging in Financial Portfolios and Variable Annuities}

        \vspace*{1.0cm}

        \normalsize
        by \\

        \vspace*{1.0cm}

        \Large
        Xintong Li \\

        \vspace*{3.0cm}

        \normalsize
        A thesis \\
        presented to the University of Waterloo \\ 
        in fulfillment of the \\
        thesis requirement for the degree of \\
        Doctor of Philosophy \\
        in \\
        Actuarial Science \\

        \vspace*{2.0cm}

        Waterloo, Ontario, Canada, 2022 \\

        \vspace*{1.0cm}

        \copyright\ Xintong Li, 2022 \\
        \end{center}
\end{titlepage}

% The rest of the front pages should contain no headers and be numbered using Roman numerals starting with `ii'
\pagestyle{plain}
\setcounter{page}{2}

\cleardoublepage % Ends the current page and causes all figures and tables that have so far appeared in the input to be printed.
% In a two-sided printing style, it also makes the next page a right-hand (odd-numbered) page, producing a blank page if necessary.

 
% E X A M I N I N G   C O M M I T T E E (Required for Ph.D. theses only)
% Remove or comment out the lines below to remove this page
\begin{center}\textbf{Examining Committee Membership}\end{center}
  \noindent
The following served on the Examining Committee for this thesis. The decision of the Examining Committee is by majority vote.
  \bigskip
  
%   \noindent
% \begin{tabbing}
% Internal-External Member: \=  \kill % using longest text to define tab length
% External Examiner: \>  John Smith \\ 
% \> Professor, Dept. of Philosophy of XX, University of ABC \\
% \end{tabbing} 
%   \bigskip
  
  \noindent
\begin{tabbing}
Internal-External Member: \=  \kill % using longest text to define tab length
Supervisor(s): \> Mingbin Feng \\
\> Assistant Professor, Dept. of Statistics and Actuarial Science\\
\> University of Waterloo \\[1cm]
\> Tony S. Wirjanto \\
\> Professor, Dept. of Statistics and Actuarial Science \\
\> University of Waterloo \\
\end{tabbing}
  \bigskip
  
  \noindent
  \begin{tabbing}
Internal-External Member: \=  \kill % using longest text to define tab length
Internal Member: \> Mary R. Hardy \\
\> Professor, Dept. of Statistics and Actuarial Science \\
\> University of Waterloo\\[1cm]
\> Chengguo Weng \\
\> Professor, Dept. of Statistics and Actuarial Science \\
\> University of Waterloo \\
\end{tabbing}
  \bigskip
  
%   \noindent
% \begin{tabbing}
% Internal-External Member: \=  \kill % using longest text to define tab length
% Internal-External Member: \> John Smith \\
% \> Professor, Dept. of XX, University of ABC \\
% \end{tabbing}
%   \bigskip
  
%   \noindent
% \begin{tabbing}
% Internal-External Member: \=  \kill % using longest text to define tab length
% Other Member(s): \> John Smith \\
% \> Professor, Dept. of XX, University of ABC \\
% \end{tabbing}

\cleardoublepage

% D E C L A R A T I O N   P A G E
% -------------------------------
  % The following is a sample Delaration Page as provided by the GSO
  % December 13th, 2006.  It is designed for an electronic thesis.
  \noindent
I hereby declare that I am the sole author of this thesis. This is a true copy of the thesis, including any required final revisions, as accepted by my examiners.

  \bigskip
  
  \noindent
I understand that my thesis may be made electronically available to the public.

\cleardoublepage

% A B S T R A C T
% ---------------

\begin{center}\textbf{Abstract}\end{center}

Quantitative risk management of modern financial derivatives and actuarial products is intricate and often requires modelling the underlying stochasticity with Monte Carlo simulations.
A simulation scheme is flexible, and it can easily adapt to changes in model assumptions and market conditions.
However, as multiple sources of risk are considered for long time horizons, such simulation schemes become expensive to implement. 
Tremendous research effort is dedicated in the design of computationally efficient machine learning-based procedures that mitigate the computational burden of a standard simulation procedure.
In machine learning, model flexibility comes at the expense of model resilience, which is especially crucial for risk management tasks.
This study considers the management of tail risk for complex financial and actuarial products with resilient machine learning-based nested simulation procedures and reinforcement learning-based hedging policies.
Our approaches offer significant improvements over a standard simulation procedures and illustrate a resilient machine learning model is built by accounting for problem complexity and data structure, quality, and quantity.


\cleardoublepage

% A C K N O W L E D G E M E N T S
% -------------------------------

\begin{center}\textbf{Acknowledgements}\end{center}

I would like to thank Professor Mingbin Feng and Professor Tony Wirjanto for their valuable support. They have patiently helped me through my journey as a Ph.D. student.  
\cleardoublepage

% T A B L E   O F   C O N T E N T S
% ---------------------------------
\renewcommand\contentsname{Table of Contents}
\tableofcontents
\cleardoublepage
\phantomsection    % allows hyperref to link to the correct page

% L I S T   O F   T A B L E S
% ---------------------------
\addcontentsline{toc}{chapter}{List of Tables}
\listoftables
\cleardoublepage
\phantomsection		% allows hyperref to link to the correct page

% L I S T   O F   F I G U R E S
% -----------------------------
\addcontentsline{toc}{chapter}{List of Figures}
\listoffigures
\cleardoublepage
\phantomsection		% allows hyperref to link to the correct page

% GLOSSARIES (Lists of definitions, abbreviations, symbols, etc. provided by the glossaries-extra package)
% -----------------------------
\printglossaries
\cleardoublepage
\phantomsection		% allows hyperref to link to the correct page

% Change page numbering back to Arabic numerals
\pagenumbering{arabic}

