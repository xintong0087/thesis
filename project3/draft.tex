\chapter{Transfer Learning for Rapid Adaptation of Deep Neural Network Metamodels in Dynamic Hedging of Variable Annuities} \label{chap:project3}

\section{Introduction}

In the evolving landscape of financial markets, insurance products such as VAs have gained significant interest due to their ability to provide both investment growth and guaranteed benefits. 
Managing the risks associated with these products, especially in volatile market conditions, is a complex task that demands sophisticated financial modeling techniques. 
Traditional machine learning models often struggle to capture the intricate nonlinear relationships and temporal dependencies inherent in financial data.
To address the computational burden, metamodeling techniques have been proposed, where a simpler model approximates the outcomes of the more complex simulation. 
In particular, deep neural networks (DNNs), and specifically LSTM networks, have been employed as metamodels to predict the outcomes of the inner simulations efficiently
Chapter~\ref{chap:project2} introduces a nested simulation framework for dynamic hedging of VAs, where the inner simulation is approximated by an LSTM metamodel.
It shows that crude RNN and LSTMs are well-suited for metamodeling Monte Carlo simulation of financial time series and can capture the long-term dependencies in the data.

Despite the advantages of using DNN metamodels, a significant challenge arises when market conditions change or new VA contracts with different features are introduced. 
Retraining neural network metamodels from scratch in response to every change is computationally inefficient and time-consuming.
Moreover, financial markets are inherently dynamic, with frequent shifts in volatility, interest rates, and other risk factors~\citep{cont2001empirical}.
Therefore, it is essential to develop methods that can rapidly adapt existing metamodels to new conditions without incurring the full computational cost of retraining.
Transfer Learning offers a compelling solution to this problem by enabling the reuse of a pre-trained model on a new but related task~\citep{pan2009survey}.
In the context of DNNs, transfer learning involves leveraging the knowledge acquired during training on one dataset to improve learning performance on a different dataset~\citep{yosinski2014transferable}.
This approach can significantly reduce training time and computational resources while enhancing model generalization.
Instead of starting from scratch, a new DNN metamodel can be built on top of the pre-trained model and fine-tuned on the new data, allowing it to adapt quickly to changing market conditions and new VA contracts.

In this chapter, we explore the application of transfer learning to the dynamic hedging of VAs using RNN and LSTM metamodels.
We propose a novel transfer learning framework that accelerates the training of DNN metamodels for nested simulation in a dynamic hedging problem.
Our approach involves pre-training an LSTM network on a large dataset of VA simulations and then fine-tuning it on a smaller dataset of new VA contracts.
We evaluate the performance of the transfer learning framework on a real-world dataset of VA contracts and compare it with training the LSTM network from scratch.

The rest of this chapter is organized as follows.
Section~\ref{sec3:background} provides an overview of the dynamic hedging problem for VAs and the use of LSTM networks as metamodels in a nested simulation procedure.
Section~\ref{sec3:transfer_learning} introduces the transfer learning framework for rapid adaptation of LSTM metamodels in dynamic hedging.
Section~\ref{sec3:experiments} presents the experimental setup and results, comparing the performance of transfer learning with training from scratch.
Finally, Section~\ref{sec3:conclusion} concludes the chapter and discusses future research directions.

\section{Background} \label{sec3:background}

\section{Transfer Learning for Rapid Adaptation of LSTM Metamodels} \label{sec3:transfer_learning}



\section{Numerical Experiments} \label{sec3:experiments}

\section{Conclusion} \label{sec3:conclusion}

